\documentclass[10pt,twocolumn,letterpaper]{article}
\usepackage{cvpr}
\usepackage{times}
\usepackage{epsfig}
\usepackage{graphicx}
\usepackage{amsmath}
\usepackage{amssymb}
\usepackage{times}
\usepackage{stfloats}
\usepackage[pagebackref=true,colorlinks,linkcolor=red,citecolor=green,breaklinks=true,bookmarks=false]{hyperref}
\cvprfinalcopy
\def\cvprPaperID{****} % *** Enter the CVPR Paper ID here
\def\httilde{\mbox{\tt\raisebox{-.5ex}{\symbol{126}}}}
\title{Generative Adversarial Nets}
\author{Cheng Guan\\\\
July 24, 2018}
\begin{document}
\maketitle
\section{Related Work}
Today, I continue to read the paper about Generative Adversarial Nets. After reading the paper, I know that Noise-contrastive estimation (NCE) \cite{gutmann2010noise} involves training a generative model by learning the weights that make the model useful for discriminating discriminating data from a fixed noise distribution. Using a previously trained models as distribution allows training a sequence of models of increasing quality. This can be seen as an informal competition mechanism similar in spirit to the formal competition used in the adversarial network game. The key limitation of NCE is that its ``discriminator'' is defined by the ratio of the probability densities of the noise distribution and the model distribution, and thus requires the ability to evaluate and backpropagate through both densities.
\par
Some previous work has used the general concept of having two neural networks compete. The most
relevant work is predictability minimization \cite{schmidhuber1992learning}. 
\par
In predictability minimization, each hidden unit
in a neural network is trained to be different from the output of a second network, which predicts
the value of that hidden unit given the value of all of the other hidden units. This work differs from
predictability minimization in three important ways: 1) in this work, the competition between the
networks is the sole training criterion, and is sufficient on its own to train the network. Predictability
minimization is only a regularizer that encourages the hidden units of a neural network to be statistically independent while they accomplish some other task; it is not a primary training criterion. 2) The nature of the competition is different. In predictability minimization, two networks' outputs
are compared, with one network trying to make the outputs similar and the other trying to make the
outputs different.
\section{Adversarial Nets}
The adversarial modeling framework is most straightforward to apply when the models are both
multilayer perceptrons. To learn the generator’s distribution $p_g$ over data $x$, they define a prior on
input noise variables $p_z\left(z\right)$, then represent a mapping to data space as $G\left(z;\theta_g\right)$, where $G$ is a differentiable function represented by a multilayer perceptron with parameters $\theta_g$ . The authors also define a
second multilayer perceptron $D\left(x;\theta_d\right)$ that outputs a single scalar. $D\left(x\right)$ represents the probability
that $x$ came from the data rather than $p_g$ . They train $D$ to maximize the probability of assigning the
correct label to both training examples and samples from $G$. They simultaneously train $G$ to minimize
$\log\left(1-D\left(G\left(z\right)\right)\right)$. In other words, $D$ and $G$ play the following two-player minimax game with
value function $V\left(G,D\right)$ as shown in Eq.\ref{eq1}:
\begin{equation}
  %\begin{aligned}
 \min_G \max_D \mathbb{E}_{x\sim P_{data}\left(x\right)} \log\left[D\left(x\right)\right] +\\
 \mathbb{E}_{z\sim P_{z}\left(z\right)} \log\left[1-D\left(G\left(z\right)\right)\right]
%\end{aligned}
\label{eq1}
\end{equation}
In the next section, the authors present a theoretical analysis of adversarial nets, essentially showing that
the training criterion allows one to recover the data generating distribution as $G$ and $D$ are given
enough capacity, \textit{i.e.}, in the non-parametric limit. See Fig.~\ref{fig1} for a less formal, more pedagogical
explanation of the approach. In practice, they must implement the game using an iterative, numerical
approach. Optimizing $D$ to completion in the inner loop of training is computationally prohibitive,
and on finite datasets would result in overfitting.
\begin{figure*}[ht]
	\centering
	\includegraphics[width=1\linewidth]{1.png}
	\label{fig1}
	\caption{(a)
		Consider an adversarial pair near convergence: $p_g$ is similar to p data and $D$ is a partially accurate classifier.
		(b) In the inner loop of the algorithm D is trained to discriminate samples from data, converging to $D^\ast\left(x\right)=\frac{P_{data}\left(x\right)}{P_{data}\left(x\right)+P_g\left(x\right)}$
		. (c) After an update to $G$, gradient of $D$ has guided $G\left(z\right)$ to flow to regions that are more likely
		to be classified as data. (d) After several steps of training, if $G$ and $D$ have enough capacity, they will reach a
		point at which both cannot improve because $p_g = p_{data}$.}
\end{figure*}
\section{Theoretical Results}
The generator $G$ implicitly defines a probability distribution $p_g$ as the distribution of the samples
$G\left(z\right)$ obtained when $z \sim p_z$ . Therefore, They would like algorithm to converge to a good estimator
of p data , if given enough capacity and training time. The results of this section are done in a non-parametric setting, \textit{e.g.} they represent a model with infinite capacity by studying convergence in the space of probability density functions.
\par
In practice, Eq.\ref{eq1} may not provide sufficient gradient for $G$ to learn well. Early in learning,
when $G$ is poor, $D$ can reject samples with high confidence because they are clearly different from
the training data. In this case, $\log\left(1-D\left(G\left(z\right)\right)\right)$ saturates. Rather than training $G$ to minimize
$\log\left(1-D\left(G\left(z\right)\right)\right)$ they can train $G$ to maximize $\log D\left(D\left(G\left(z\right)\right)\right)$. This objective function results in the
same fixed point of the dynamics of $G$ and $D$ but provides much stronger gradients early in learning.
{\small
	\bibliographystyle{ieee}
	\bibliography{GAN6}
}
\end{document}
\end{document}
\documentclass[10pt,twocolumn,letterpaper]{article}
\usepackage{cvpr}
\usepackage{times}
\usepackage{epsfig}
\usepackage{graphicx}
\usepackage{amsmath}
\usepackage{amssymb}
\usepackage{times}
\usepackage{stfloats}
\usepackage[pagebackref=true,colorlinks,linkcolor=red,citecolor=green,breaklinks=true,bookmarks=false]{hyperref}
\cvprfinalcopy
\def\cvprPaperID{****} % *** Enter the CVPR Paper ID here
\def\httilde{\mbox{\tt\raisebox{-.5ex}{\symbol{126}}}}
\title{Look at the Driver, Look at the Road: No Distraction! No Accident!}
\author{Cheng Guan\\\\
July 8, 2018}
\begin{document}
\maketitle
\section{Archetypes and Archetype Hull}
According to this paper, I know that archetype was originally called psychology \cite{jung2014archetypes} and was used to describe a very typical and generally understood paradigm of a set of objects. Prototype analysis aims to find a compact ``pure type'', \emph{i.e.}, the so-called archetypes, Thus, the typical pattern of this set of objects is covered in the archetypes, while the other objects are just a combination of simulations or archetypes.
\subsection{Concept of Archetypes}
The concept of archetypes actually is not far away from us, which exists in a variety of
subjects including literature, philosophy, and psychology
marketing \cite{li2003archetypal} and statistics\cite{eugster2011weighted}. The idea of archetypes is
recently introduced into pattern recognition and informatics
\cite{thurau2010yes,seiler2013archetypal}. Understanding the concept of archetypes
in face recognition is intuitive: people can easily remember
someone with a very distinctive facial appearance; some
people are thought of being very similar to a few distinctive
faces. Such phenomena serve as the evidence of identifying
archetypes and correlating unknown faces to known
archetypes in a recognition process.
\subsection{Archetype Seeking}
The time efficiency of finding archetypes is an important
concern. It is known that for an input data set $\chi$ of
$n$ points with d dimensions, the time complexity for computing
a convex hull enclosing $\chi$ is as high as 
\cite{thurau2010nearest,thurau2010yes}. For a set of tens of thousands of images with
high-dimensional descriptors, exactly solving this convex
hull problem quickly becomes computationally intractable.
To this end, a few algorithms have been designed to achieve
approximate solutions, of which the simplex volume maximization
algorithm \cite{thurau2010yes} can provide a good approximate
solution in a linear time complexity $O\left(n\right)$.
\par
With a mild assumption that all simplex vertices
are equidistant, the Cayley-Menger determinant gives
a simplified volume formula as Eq.\ref{eq1}
\begin{equation}
\begin{aligned}
V_{ol}{\left(S\right)}^2=&\frac{a^{2\left(m-1\right)}}{2^{m-1}{\left(\left(m-1\right)!\right)}^2} 
\left( {\frac{2}{a^4}\sum_{j=1}^{m-1}\sum_{j^\prime=j+1}^{m-1}d_{j,m}^2 d_{j^\prime,m}^2} \right. \\& + \left.{\frac{2}{a^2}\sum_{j=1}^{m-1}d_{j,m}^2 - \frac{m-2}{a^4}\sum_{j=1}^{m-1}d_{j,m}^4-\left(m-2\right)}\right)
\end{aligned}
\label{eq1}
\end{equation}
\section{Archetype Hull Ranking}
In this section, the authors address the problem that they raised in
Section 3: how to correlate an input sample with archetypes.
Note that this sample may be an unseen sample outside the
set of available training samples, and that these archetypes
can be stored in memory for real-time computations.
\par
Since the archetype hull encloses most training samples in $\chi$ , we can project any input sample $x \in \mathbb{R}^d$ onto S, thereby obtaining a new (lossy)
representation of $x$ in terms of the stored archetypes in $\upsilon$.
Formally, they reconstruct the input $x$ using a convex combination
of the archetypes in $\upsilon$ as Eq.\ref{eq2}
\begin{equation}
\begin{aligned}
\min_{z\left(x\right)\in\mathbb{R}^m}\left \| x-Uz\left(x\right) \right \| \\
 z\left(x\right) \geq 0, \quad \bf{1}^\top z\left(x\right)=1
\end{aligned}
\label{eq2}
\end{equation}

{\small
\bibliographystyle{ieee}
\bibliography{FD}
}
\end{document}
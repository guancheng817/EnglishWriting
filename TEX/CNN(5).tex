\documentclass[a4paper,twocolumn]{article}
\usepackage{indentfirst,graphicx,float}
\usepackage{balance}
\usepackage{cite}
\usepackage{amsmath}
\bibliographystyle{plain}
\setlength{\parindent}{2em}
\linespread{1.5}
\title{Convolutional layer-Convolution}
\author{Cheng Guan}
\date{May 13, 2018}
\begin{document}
\maketitle
\balance
\section{Convolution}
 Let off the concept of convolution first. For simplicity,
consider an image of 5 x 5 and a 3 x 3 convolution kernel.
There are 9 parameters in the convolution kernel here.
In this case, the convolution kernel actually has 9 neurons,
and their output forms a 3 x 3 matrix, called the feature graph.
The first neuron is connected to the first 3 x 3 part of the image, and the second neuron is connected to second part.
As shown in the following figure ~\ref{fig1}.

\begin{figure}[htbp]
\centering
\includegraphics[width=6cm,height=5cm]{1.png}
\caption{Convolution Process}
\label{fig1}
\end{figure}

Above the graph is the output of the first neuron, and below is the output of the second neurons.
The formula of each neuron is :\\
\begin{center}
    \begin{equation}
     f(x)=act(\sum_{i,j}^n\theta_{(n-i)(n-j)}x_{ij}+b)
    \end{equation}
\end{center}

Now let's recollect the discrete convolution operation.
Assuming that there are two dimensional discrete functions f (x, y) and g (x, y),
their convolution is defined as:\\
\begin{center}
    \begin{equation}
     f(m,n)*g(m,n)=\sum_u^{\infty}\sum_v^{\infty}f(u,v)g(m-u,n-v)
    \end{equation}
\end{center}
The 9 neurons in the above example are actually equivalent to the convolution operation of the image and convolution kernel
after the output of the neurons is completed.\cite{test2}

\bibliography{cite20}
\end{document}

\documentclass[10pt,twocolumn,letterpaper]{article}
\usepackage{cvpr}
\usepackage{times}
\usepackage{epsfig}
\usepackage{graphicx}
\usepackage{amsmath}
\usepackage{amssymb}
\usepackage{times}
\usepackage{stfloats}
\usepackage[pagebackref=true,colorlinks,linkcolor=red,citecolor=green,breaklinks=true,bookmarks=false]{hyperref}
\cvprfinalcopy
\def\cvprPaperID{****} % *** Enter the CVPR Paper ID here
\def\httilde{\mbox{\tt\raisebox{-.5ex}{\symbol{126}}}}
\title{Learning Face Age Progression: A Pyramid Architecture of GANs}
\author{Cheng Guan\\\\
July 18, 2018}
\begin{document}
\maketitle
\begin{abstract}
As we know, the two underlying requirements of face age
progression, \textit{i.e.} aging accuracy and 
identity permanence, are not studied in the recent 
research. In this paper, the authors proposed a new
generative adversarial network. It models the 
constraints for intrinsic subject-specific 
characteristics and the age-specific facial changes, 
ensuring that the generated faces present desired aging 
effect while keeping personalized properties stable. 
In order to generate more facial details, high-level 
age-specific features conveyed by the synthesized face 
are estimated by a pyramidal adversarial discriminator 
at multiple scales, which simulates the aging effects in a finer manner. The method is applied to various face samples and vivid aging effects are achieved. 
\end{abstract}
%
\section{Introduction}
To the best of our knowledge, age progression is the process of rendering a given face image to present the effects of aging. It is often used in entertainment 
industry and forensics, \textit{e.g.}, forecasting facial appearances of young children when they grow up 
or generating contemporary photos for missing persons.
The intrinsic complexity of physical aging, the interferences caused by other factors (\textit{e.g.}, PIE variations), and shortage of labeled aging data  make face age progression a rather difficult problem.
\par
The recent research have made great progress tackling this issue, where aging accuracy and identity permanence are regarded as the two underlying premises of its success \cite{suo2010compositional,shu2015personalized,yang2016face}. The previous methods either works in a difficulty manner or limit the diversity of aging 
patterns. The deep generation networks have demonstrated a good capability in image generation \cite{dosovitskiy2016generating,goodfellow2014generative,isola2017image} and have also been investigated for age progression \cite{wang2016recurrent,zhang2017age}.
\par
In this paper, the authors propose a new approach to face age progression, which integrates the advantage of
GAN in synthesizing visually plausible images with 
prior domain knowledge in human aging. The method proposed uses Convolutional Neutral Networks (CNN) based
generator to learn age transformation, and it separately models different face attributes depending on their changes over time. The author emphasize that synthesis of the entire face is important since the parts of forehead and hair also significantly impact the perceived age. To achieve this and further enhance the aging details, their method use the intrinsic hierarchy of deep networks, and a discriminator of the pyramid architecture is designed to estimate
high-level age-related clues in a fine-grained way.
more photorealistic imageries are generated (see Fig.~\ref{fig1} for an illustration of aging results).
\begin{figure}[t]
	\centering
	\includegraphics[width=1\linewidth]{1.png}
	\caption{Demonstration of aging simulation results (images in the first column are input faces of two subjects).}
	\label{fig1}
\end{figure}
\section{Related Work}
The authors make use of the image generation ability of GAN, 
and presents a different but effective method,
where the age-related GAN loss is adopted for age transformation, 
the individual-dependent critic is used to keep the identity cue stable, 
and a multi-pathway discriminator is applied 
to refine aging detail generation. This solution is more
powerful in dealing with the core issues of age progression,
\textit{i.e.} age accuracy and identity preservation.
\section{Method}
\subsection{Overview}
A classic GAN contains a generator $G$ and a discriminator $D$, which iteratively trained via an adversarial process. The generative function G tries to capture the underlying data density and confuse the discriminative function $D$, while the optimization procedure of $D$ aims to achieve
the distinguishability and distinguish the natural face images from the fake ones generated by G. Both $G$ and $D$ can be approximated by neural networks, e.g., Multi-Layer Perceptron (MLP). The risk function of optimizing this mini-max two-player game can be written as Eq.\ref{eq1}:
\begin{equation}
  \begin{aligned}
   \mathcal{V}\left(D,G\right) = &\min_G \max_D \mathbb{E}_{x\sim P_{data}\left(x\right)} \log\left[D\left(x\right)\right] +\\
   & \mathbb{E}_{z\sim P_{z}\left(z\right)} \log\left[1-D\left(G\left(z\right)\right)\right]
  \end{aligned}
  \label{eq1}
\end{equation}
In research of this paper, the CNN based generator takes young
faces as inputs, and learns a mapping to a domain corresponding 
to elderly faces.
{\small
	\bibliographystyle{ieee}
	\bibliography{GNN3}
}
\end{document}
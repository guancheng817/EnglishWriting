\documentclass[a4paper,twocolumn]{article}
\usepackage{indentfirst,graphicx,float}
\usepackage{balance}
\usepackage{cite}
\usepackage{amsmath}
\bibliographystyle{plain}
\setlength{\parindent}{2em}
\linespread{1.5}
\title{Spatial Arrangement}
\author{Cheng Guan}
\date{May 11, 2018}
\begin{document}
\maketitle
\balance
\section{Spatial Arrangement}
The size of an output unit is controlled by the following three variables: \textbf{depth, stride and zero-padding}.\\
\textbf{Depth}: It controls the depth of the output unit, that is, the number of filter, and the number of neurons connected to the same area. Also named: depth column.\\
\textbf{Stride}: It controls the distance between two adjacent implicit units at the same depth and the input regions connected to them. If the stride is very small (for example, stride = 1), the overlap area of the input area of the adjacent hidden units will be large.\\
\textbf{Zero-padding}: We can change the size of the input unit by adding zeros around the input unit to control the size of the output unit.\\
Let's first define a few symbols:\\
\textbf{W}:The size of the input unit (wide or high)\\
\textbf{F}:Receptive field\\
\textbf{S}:Stride \\
\textbf{P}:Number of zero-padding\\
\textbf{K}:Depth of output units\\

We can use the following formula to calculate a hidden unit in a single output unit of a dimension (width or height):
\begin{center}
    \begin{equation}
    \frac{W-F+2P}{S}+1
    \end{equation}
\end{center}

If the calculation result is not an integer,
it means that the existing parameters are not suitable for input, stride is not set, or zero is needed.Here is an example to illustrate ~\ref{fig1}.
This is an example of one dimension. The input unit of the left model has 5,
that is, W=5, the boundary each complement one zero, that is, P=1, the stride is 1,
that is, S=1, the receptive field is 3,
because each output hidden unit connects 3 input units, that is, F=3,
and the number of the output hidden units can be calculated according to the above formula: $\frac{5-3+2}{1}+1=5$.
The model on the right is to change the stride to 2, and the rest remains unchanged.
It can be calculated that the output size is $\frac{5-3+2}{2}+1=3$, which is also consistent with the illustration.
If the stride is changed to 3,
the formula can not be divisible, which means that the step size of 3 can not coincide with the size of the input unit.\cite{test2}

\begin{figure*}[htbp]
\centering
\includegraphics[scale=0.4]{1.png}
\caption{Example of Illustration}
\label{fig1}
\end{figure*}

\bibliography{cite20}
\end{document}

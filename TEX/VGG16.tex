\documentclass[10pt,twocolumn,letterpaper]{article}
\usepackage{cvpr}
\usepackage{times}
\usepackage{epsfig}
\usepackage{graphicx}
\usepackage{amsmath}
\usepackage{amssymb}
\usepackage{times}
\usepackage{stfloats}
\usepackage[pagebackref=true,colorlinks,linkcolor=red,citecolor=green,breaklinks=true,bookmarks=false]{hyperref}
\cvprfinalcopy
\def\cvprPaperID{****} % *** Enter the CVPR Paper ID here
\def\httilde{\mbox{\tt\raisebox{-.5ex}{\symbol{126}}}}
\title{Very Deep Convolutional Networks for Large-Scale Image Recognition}
\author{Cheng Guan\\\\
August 9, 2018}
\begin{document}
\maketitle
\begin{abstract}
	Today, after I know about deep learning deeper, I want to learn some classic convolutional network. 
	This is a paper on VGG16. In this work the authors investigate the effect of the convolutional network depth on its
	accuracy in the large-scale image recognition setting. Their main contribution is
	a thorough evaluation of networks of increasing depth using an architecture with
	very small (3 $\times$ 3) convolution filters, which shows that a significant improvement
	on the prior-art configurations can be achieved by pushing the depth to 16–19
	weight layers. These findings were the basis of their ImageNet Challenge 2014
	submission, where their team secured the first and the second places in the localisation
	 and classification tracks respectively. They also show that their representations
	generalise well to other datasets, where they achieve state-of-the-art results. They
	have made two best-performing ConvNet models publicly available to facilitate 
	further research on the use of deep visual representations in computer vision.
\end{abstract}
\section{Introduction}
Convolutional networks (ConvNets) have recently enjoyed a great success in large-scale image and video recognition \cite{krizhevsky2012imagenet,zeiler2014visualizing,sermanet2013overfeat,simonyan2014two} which has become possible due to the large public image repositories, such as ImageNet, and high-performance computing systems, such as GPUs
or large-scale distributed clusters. In particular, an important role in the advance
of deep visual recognition architectures has been played by the ImageNet Large-Scale Visual Recognition Challenge (ILSVRC) \cite{russakovsky2015imagenet}, which has served as a testbed for a few
generations of large-scale image classification systems, from high-dimensional shallow feature encodings (the winner of ILSVRC-2011) to deep ConvNets (the winner of ILSVRC-2012).
\par
With ConvNets becoming more of a commodity in the computer vision field, a number of attempts have been made to improve the original architecture of \cite{krizhevsky2012imagenet} in a
bid to achieve better accuracy. For instance, the best-performing submissions to the ILSVRC-
2013 utilized smaller receptive window size and
smaller stride of the first convolutional layer. Another line of improvements dealt with training
and testing the networks densely over the whole image and over multiple scales.
\par
The authors come up with significantly more accurate ConvNet architectures, which not only
achieve the state-of-the-art accuracy on ILSVRC classification and localisation tasks, but are also
applicable to other image recognition datasets, where they achieve excellent performance even when
used as a part of a relatively simple pipelines
\section{Convnet Configuration}
To measure the improvement brought by the increased ConvNet depth in a fair setting, all their
ConvNet layer configurations are designed using the same principles, inspired by \cite{ciresan2011flexible,krizhevsky2012imagenet}. 
In this section, the authors first describe a generic layout of their ConvNet
configurations.
\subsection{Architecture}
During training, the input to their ConvNets is a fixed-size 224 $\times$ 224 RGB image. The only pre-processing they do is subtracting the mean RGB value, computed on the training set, from each pixel.
The image is passed through a stack of convolutional (conv.) layers, where they use filters with a very
small receptive field: 3 $\times$ 3 (which is the smallest size to capture the notion of left/right, up/down,
center). In one of the configurations they also utilize 1 $\times$ 1 convolution filters, which can be seen as
a linear transformation of the input channels (followed by non-linearity). The convolution stride is
fixed to 1 pixel; the spatial padding of conv. layer input is such that the spatial resolution is preserved
after convolution, i.e. the padding is 1 pixel for 3 $\times$ 3 conv. layers. Spatial pooling is carried out by
five max-pooling layers, which follow some of the conv. layers (not all the conv. layers are followed
by max-pooling). Max-pooling is performed over a 2 $\times$ 2 pixel window, with stride 2.
\par
A stack of convolutional layers (which has a different depth in different architectures) is followed by
three Fully-Connected (FC) layers: the first two have 4096 channels each, the third performs 1000-way ILSVRC classification and thus contains 1000 channels (one for each class). The final layer is
the softmax layer. The configuration of the fully connected layers is the same in all networks.
\bibliography{VGG}
\bibliographystyle{ieee}
\end{document}

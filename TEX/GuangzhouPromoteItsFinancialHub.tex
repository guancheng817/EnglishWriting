\documentclass[a4paper,twocolumn]{article}
\usepackage{indentfirst,graphicx,float}
\usepackage{balance}
\usepackage{cite}
\bibliographystyle{plain}
\setlength{\parindent}{2em}
\linespread{1.5}
\title{In Silicon Valley, Guangzhou Promotes Its Financial Hub}
\author{Cheng Guan}
\date{\today}
\begin{document}
\maketitle
\balance
A delegation of Chinese government officials, bankers and scholars shared the city's financial policies,
development and strategies with local officials and companies at the 2018 Guangzhou
Financial Roadshow to Global Bay Areas on Wednesday in Palo Alto, California.

Silicon Valley is the second stop of the weeklong road show,
which started on May 14 in Tokyo and will conclude on May 21 in New YorkAs the following picture ~\ref{fig1} shows..

\begin{figure}[H]
\centering
\includegraphics[width=7cm,height=4cm]{1.png}
\caption{Guangzhou Promotes Its Financial Hub}
\label{fig1}
\end{figure}

"It's the first time Guangzhou has promoted its financial drive in the US.
We hope to learn from Silicon Valley's experience, attract financial talent and
then explore the potential of cooperation with Silicon Valley," said Chen Ping,
deputy director of the Guangzhou Financial Affairs Bureau, organizer of the event\cite{test2}.

To attract overseas financial institutions,
the city offers any newly established or settled financial institution an award of up to 3.9 million based on the registered capital.

A lump-sum award of 314,000 will be given to the regional head office of the bank, securities firm or insurance company. The city also has similar incentive policies for overseas talent specializing in financial technologies, investment, risk management and other key areas.
"The Pearl River Delta is the fastest-growing area in the entire world.

If you look at the scale of population and academic focus, 2.8 million students graduate from technologies each year in China, five times that of the US,"
said James Shea, senior vice-president of UBS Financial Services.

\bibliography{cite2}
\end{document}

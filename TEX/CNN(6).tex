\documentclass[a4paper,twocolumn]{article}
\usepackage{indentfirst,graphicx,float}
\usepackage{balance}
\usepackage{cite}
\usepackage{amsmath}
\usepackage{balance}
\usepackage{enumerate}
\bibliographystyle{plain}
\setlength{\parindent}{2em}
\linespread{1.5}
\title{Pooling Layer}
\author{Cheng Guan}
\date{\today}
\begin{document}
\maketitle

\section{The Summary on Convolution Layer}
\begin{enumerate}[1]\setlength{\itemsep}{0pt}
 \item The input of three dimension:
   \begin{center}
      \begin{equation}
      W_1 * H_1 * D_1
      \end{equation}
  \end{center}
 \item 4 parameters need to be given:\\
     * Number of filters K\\
     * Their spatial extent F\\
     * The stride S\\
     * The amount of zero padding P
 \item The output of a three-dimensional unit $W_2 * H_2 * D_2$, which:
  \begin{center}
      \begin{equation}
      W_2=\frac{W_1-F+2P}{S}+1
      \end{equation}
      \begin{equation}
      H_2=\frac{H_1-F+2P}{S}+1
      \end{equation}
      \begin{equation}
      D_2=K
      \end{equation}
  \end{center}
  \end{enumerate}
  \section{Pooling Layer}
  Pool (downsamples) is designed to reduce the feature map.
   The pool operation is independent of each slice of depth,
   and the scale is generally 2 * 2.The operation of pooling layer is the following\cite{test2}:

    * Max Pooling. Take the maximum of 4 points. This is the most commonly used pooling method.

    * Mean Pooling. Take the mean of the 4 points.

    * Gauss pooling. Draw on the method of Gauss's vagueness. It's not common.

  The most common pooling layer is 2*2 with the stride of 2, and each depth slice of the input is sampled below.
  Each MAX operation is performed on four numbers, as shown in the following figure ~\ref{fig1}:
  \begin{figure}[htbp]
  \centering
  \includegraphics[scale=0.2]{1.png}
  \caption{Example of Pooling}
  \label{fig1}
  \end{figure}
  The pool operation will keep the size of depth .
  If the size of the input unit of the pooling layer is not an integer multiple of two,
  the zero-padding is generally used to fill the multiple of 2 and then pool.
  \bibliography{cite2}
  \end{document}

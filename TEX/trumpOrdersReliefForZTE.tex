\documentclass[a4paper,twocolumn]{article}
\usepackage{indentfirst,graphicx,float}
\usepackage{balance}
\usepackage{cite}
\bibliographystyle{plain}
\setlength{\parindent}{2em}
\linespread{1.5}
\title{Trump Orders Relief For ZTE}
\author{Cheng Guan}
\date{\today}
\begin{document}
\maketitle
\balance
US President Donald Trump said on Sunday that he has instructed the Commerce Department
to help China's ZTE Corp get back to business, a sign that experts believe
helps set a positive tone for trade talks in Washington between senior Chinese and US officials.

ZTE, one of the largest telecom-equipment manufacturers in China,
announced last week that it would suspend major operations after the US Commerce Department
last month banned US companies from selling to the Chinese company until 2025.

The US has charged ZTE with violating US laws by illegally shipping US goods to Iran and
breaching a deal reached last year. ZTE appealed the US ban. ZTE argues that
the ban "will severely impact the survival and development" of the company,
which depends on US companies providing about a quarter of its technology components.

In a tweet on Sunday, Trump said, "President Xi, of China, and I are working together to give massive Chinese phone company, ZTE, a way to get back into business, fast.
Too many jobs in China lost. Commerce Department has been instructed to get it done!"\cite{test2}
As the following picture ~\ref{fig1} shows.
\begin{figure}[H]
\centering
\includegraphics[width=7cm,height=4cm]{1.png}
\caption{ZTE}
\label{fig1}
\end{figure}


ZTE and Huawei, another Chinese telecom-equipment giant,
have been successful in other parts of the world, but they have long been the target of some US lawmakers and politicians,
 who described the Chinese companies as posing a threat to US national security.

Many in China believe the US restrictions on the two companies are aimed at
disrupting the development of China's high-tech sector,
in particular the 10 industries defined in the Made in China 2025 effort to advance manufacturing capability.



\bibliography{cite20}
\end{document}

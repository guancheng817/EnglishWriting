\documentclass[10pt,twocolumn,letterpaper]{article}
\usepackage{cvpr}
\usepackage{times}
\usepackage{epsfig}
\usepackage{graphicx}
\usepackage{amsmath}
\usepackage{amssymb}
\usepackage{times}
\usepackage{stfloats}
\usepackage[pagebackref=true,colorlinks,linkcolor=red,citecolor=green,breaklinks=true,bookmarks=false]{hyperref}
\cvprfinalcopy
\def\cvprPaperID{****} % *** Enter the CVPR Paper ID here
\def\httilde{\mbox{\tt\raisebox{-.5ex}{\symbol{126}}}}
\title{Hierarchical Novelty Detection for Visual Object Recognition}
\author{Cheng Guan\\\\
August 5, 2018}
\begin{document}
\maketitle
\begin{abstract}
Deep neural networks have achieved impressive success
in large-scale visual object recognition tasks with a predefined set of classes. However, recognizing objects of novel
classes unseen during training still remains challenging.
The problem of detecting such novel classes has been addressed in the literature, but most prior works have focused
on providing simple binary or regressive decisions, e.g., the
output would be ``known,'' ``novel,'' or corresponding confidence intervals. In this paper, the authors study more informative
novelty detection schemes based on a hierarchical classification framework. For an object of a novel class, they aim
for finding its closest super class in the hierarchical taxonomy of known classes. To this end, they propose two different
approaches termed top-down and flatten methods, and their
combination as well.
\end{abstract}
\section{Introduction}
Object recognition in large-scale image datasets has
achieved impressive performance with deep convolutional
neural networks (CNNs) \cite{he2015delving,he2016deep,simonyan2014very,szegedy2015going}. The standard
CNN architectures are learned to recognize a predefined set
of classes seen during training. However, in practice, a new
type of objects could emerge (e.g., a new kind of consumer
product). Hence, it is desirable to extend the CNN architectures for detecting the novelty of an object (i.e., deciding
if the object does not match any previously trained object
classes). There have been recent efforts toward developing
efficient novelty detection methods \cite{bendale2016towards}, but
most of the existing methods measure only the model uncertainty, i.e., confidence score, which is often too ambiguous
for practical use. For example, suppose one trains a classifier on an animal image dataset as in Fig.~\ref{fig1}.
A standard novelty detection method can be applied to a cat-like image to evaluate its novelty, but such a method would not tell
whether the novel object is a new species of cat unseen in
the training set or a new animal species.
%
\begin{figure}
\centering
\includegraphics[width=1\linewidth]{1.png}
\caption{An illustration of their proposed hierarchical novelty detection task. In contrast to prior novelty detection
	works, the authors aim to find the most specific class label of a novel
	data on the taxonomy built with known classes.}
\label{fig1}
\end{figure}
%
\par
To address this issue, the authors design a new classification
framework for more informative novelty detection by utilizing a hierarchical taxonomy, where the taxonomy can
be extracted from the natural language information, e.g.,
WordNet hierarchy \cite{miller1995wordnet}. Their approach is also motivated by
a strong empirical correlation between hierarchical semantic relationships and the visual appearance of objects.
For example, as illustrated in Fig.~\ref{fig1}, Their goal is to distinguish ``new cat,'' ``new dog,'' and ``new animal,'' which
cannot be achieved in the standard novelty detection tasks.
they call this problem \textit{hierarchical novelty detection} task.
\par
In contrast to standard object recognition tasks with a
closed set of classes, Their proposed framework can be useful
for extending the domain of classes to an open set with taxonomy information (i.e., dealing with any objects unseen in training). In practical application scenarios, Their framework
can be potentially useful for automatically or interactively
organizing a customized taxonomy (e.g., company’s product catalog, wildlife monitoring, personal photo library) by suggesting closest categories for an image from novel categories (e.g., new consumer products, unregistered animal
species, untagged scenes or places).
\section{Related Work}
\noindent \textbf{Novelty detection.} For robust prediction, it is desirable to
detect a test sample if it looks unusual or significantly differs from the representative training data. Novelty detection
is a task recognizing such abnormality of data A
confidence score about novelty can be measured by taking
the maximum predicted probability , ensembling such
outputs from multiple models , or synthesizing a score
based on the predicted categorical distribution.
\par
\noindent \textbf{Object recognition with taxonomy.} Incorporating the hi-
erarchical taxonomy for object classification has been investigated in the literature, either to improve classification
performance, or to extend the classification tasks to
obtain more informative results \cite{deng2012hedging}.
\par
\noindent \textbf{Generalized zero-shot learning (GZSL).} they remark that
GZSL \cite{chao2016empirical} can be thought as addressing a similar task
as theirs. While the standard ZSL tasks test classes unseen
during training only, GZSL tasks test both seen and unseen
classes such that the novelty is automatically detected if the
predicted label is not a seen class.
\bibliography{VOR}
\bibliographystyle{ieee}
\end{document}
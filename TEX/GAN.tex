\documentclass[10pt,twocolumn,letterpaper]{article}
\usepackage{cvpr}
\usepackage{times}
\usepackage{epsfig}
\usepackage{graphicx}
\usepackage{amsmath}
\usepackage{amssymb}
\usepackage{times}
\usepackage{float}
\usepackage{stfloats}
\usepackage[pagebackref=true,colorlinks,linkcolor=red,citecolor=green,breaklinks=true,bookmarks=false]{hyperref}
\cvprfinalcopy
\def\cvprPaperID{****} % *** Enter the CVPR Paper ID here
\def\httilde{\mbox{\tt\raisebox{-.5ex}{\symbol{126}}}}
\title{Finding Tiny Faces in the Wild with Generative Adversarial Networ}
\author{Cheng Guan\\\\
	July 14, 2018}
\begin{document}
\maketitle
\begin{abstract}
In the field of deep learning, GAN is a hot research direction in recent years. These days, I decide to read some papers about GAN. Over the past decades, face detection techniques have been developed rapidly, and one of  challenges is detecting small faces in unconstrained conditions. The reason is
that tiny face images are often lacking detailed information and
blurring. In this paper, the authors proposed an algorithm to directly 
generate a clear high-resolution face from a blurry small
one by adopting a generative adversarial network (GAN).
The author also introduce new training losses to guide the generator network to 
recover nice details and to promote the discriminator network 
to distinguish real vs. fake and face vs. non-face.
\end{abstract}
\begin{figure*}[hb]
	\centering
	\includegraphics[scale=0.5]{1.png}
	\caption{The detection results of tiny faces in the wild. (a) is the original low-resolution blurry face, (b) is the result of
		re-sizing directly by a bi-linear kernel, (c) is the generated image by the super-resolution method, and our result (d) is learned
		by the super-resolution ($\times$4 upscaling) and refinement network simultaneously. Best viewed in color and zoomed in.}
	\label{fig1}
\end{figure*}
\section{Introduction}
Face detection has been  studied over the past
few decades and many methods have been proposed for most different scenes. Nowadays, 
the performance of small face images is far from satisfactory. 
The main difficulty for small images detection is that small 
images lack sufficient detailed information to distinguish them 
from the similar background. Some methods \cite{chakrabarti2016neural,zhu2016cms} use 
intermediate $conv$ feature maps to represent faces at special
scales, which keeps the balance between the computation burden 
and the performance. In the generator sub-network, a super-resolution network (SRN) 
is used to up-sample small faces to a fine scale
for finding those tiny faces. SRN can improve
the quality of up-sampled images with a large upscaling factors (4$\times$ in their current implementation), 
as shown in Fig.~\ref{fig1}(b) 
and Fig.~\ref{fig1}(c). However, even with SRN, up-scaled images are unsatisfactory due to faces of low resolutions.
In the discriminator sub-network, the authors introduce a
new loss function that enforces the discriminator network to
distinguish the real/fake face and face/non-face simultaneously.
%
\section{Related Work}
\subsection{Generative Adversarial Networks}
The method of this paper, they use the super-resolution 
and refinement network to generate clear and nice faces with
resolution, as shown in Fig.~\ref{fig1}. Generative adversarial network (GAN) is introduced to generate realistic-looking images from random noise. GAN is applied to image generation, image editing, representation learning,
image annotation, image super-resolving and character 
transferring. In the discriminator network, the basic 
GAN \cite{goodfellow2014generative} is trained to distinguish the real and fake high resolution images.
\section{Proposed Method}
In this section, the authors introduce their method in detail and the description of the classical GAN network.
\subsection{GAN}
GAN \cite{goodfellow2014generative} learns a generative model $G$ via an adversarial
process. It trains a generator network $G$ and a discriminator network $D$ simultaneously. The training process alternately optimizes the generator and discriminator, which compete with each other. The generator $G$ is trained for generating the samples to fool the discriminator $D$, and the discriminator $D$ is trained to distinguish the real image and the fake
image from the generator.

{\small
	\bibliographystyle{ieee}
	\bibliography{GAN}
}
\end{document}


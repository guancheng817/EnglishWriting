\documentclass[10pt,twocolumn,letterpaper]{article}
\usepackage{cvpr}
\usepackage{times}
\usepackage{epsfig}
\usepackage{graphicx}
\usepackage{amsmath}
\usepackage{amssymb}
\usepackage{times}
\usepackage{stfloats}
\usepackage[pagebackref=true,colorlinks,linkcolor=red,citecolor=green,breaklinks=true,bookmarks=false]{hyperref}
\cvprfinalcopy
\def\cvprPaperID{****} % *** Enter the CVPR Paper ID here
\def\httilde{\mbox{\tt\raisebox{-.5ex}{\symbol{126}}}}
\title{Face Recognition vi a Archetype Hull Ranking}
\author{Cheng Guan\\\\
July 6, 2018}
\begin{document}
\maketitle
\begin{abstract}
In the recent years, the archetype hull model is playing an important role in
large-scale data analytics and mining, but rarely applied to
vision problems. In this paper, the authors migrated this geometric model together to solve face recognition and verification problems.
By proposing a unified prototype hull ranking framework.
Upon a scalable graph characterized by a compact
set of archetype exemplars whose convex hull encompasses
most of the training images, the proposed framework explicitly
captures the relevance between any query and the
stored archetypes, yielding a rank vector over the archetype
hull.
\end{abstract}
\section{Introduction}
The main purpose of facial analysis is to calculate robustness and effectiveness
A measure of similarity between any input facial image pairs. Such a measure is
expected to suppress intra-personal face variations due to varying expressions,
poses, and illumination conditions. Today, fast-growing facial image resources from online photo albums and social networks offer new opportunities
At the same time, it presents new challenges to existing facial treatment methods. How can they take advantage of the
gigantic amount of face information on the Web? One feasible
approach is to upgrade current face processing systems
by augmenting web-crawled face images into their training
datasets, which therefore requires the face systems to be
easy for re-training and scalable to accommodate massive
web data. 
\par
In pursuit of scalability, the authors used a small part
archetype exemplars to represent a lot of facial training
image. These archetypes form a convex shell
Contains most faces in the training set. To the best of
our knowledge, the archetype hull model has not been applied
to the face area. 
The use of archetypes along with
the produced archetype hull may open a new avenue to
make traditional facial processing methods extendable to
Large-scale face datasets. To illustrate, Fig.~\ref{fig1} showcases face
archetypes and an archetype hull to model face images. In
this paper, they seek such archetypes using an efficient simplex
volume maximization algorithm. Subsequently, they
build a scalable graph by virtue of the archetypes whose
size is much smaller than the training data size.
\begin{figure}
  \centering
  \includegraphics[scale=0.5]{1.png}\\
  \caption{One visual example showcasing an archetype hull of face
images. They view face images as points in the image space, where
the polytope composed of a few archetype faces encloses almost
all points. Any point can be represented by a convex combination
of the archetypes, and these archetypes hence form a convex hull
of the entire point set.}\label{fig1}
\end{figure}

\section{Related Work}
In the face recognition literature, a large number of subspace
methods \cite{belhumeur1996eigenfaces,moghaddam2000bayesian,wang2004dual,wang2004unified,li2005nonparametric} working
on holistic facial features have been proposed. Recently, local
facial descriptors \cite{ahonen2006face} achieved greater accuracy
gains on many benchmark datasets. The local descriptors
attempt to extract distinctive features of image textures like local micro-patterns of face shapes like LBP
\cite{ahonen2006face}. However, intra-personal variations caused by varying
expressions, poses, and illumination conditions remain a
potential obstacle to these appearance-based methods.
As mentioned before, one of the major challenges of
modern face recognition is the explosive growth of face data.
Some efficient learning methods could provide promising
solutions. For example, the LARK method \cite{lowe2004distinctive} developed
a special kind of features to obtain a training free classifier.
Lately, a scalable neighborhood graph, Anchor Graph
, was proposed to accommodate massive training samples,
and has shown excellent performance in large-scale
semi-supervised learning  and image retrieval . In
this paper, the authors employ the Anchor Graph model to deal with
large quantities of face images since it scales linearly with
the training set size in terms of both space and time complexities.

{\small
\bibliographystyle{ieee}
\bibliography{FR}
}
\end{document}

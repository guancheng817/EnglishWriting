\documentclass[10pt,twocolumn,a4paper]{article}
\usepackage{times}
\usepackage{epsfig}
\usepackage{graphicx,float}
\usepackage{indentfirst}
\usepackage{balance}
\usepackage{cite}
\usepackage[pagebackref=true,breaklinks=true,colorlinks,bookmarks=false]{hyperref}
\usepackage{geometry}
\geometry{left=2.0cm,right=2.0cm,top=2.5cm,bottom=2.5cm}
\setlength{\parindent}{1em}
\title{Global Hypothesis Generation for 6D Object Pose Estimation}
\author{Cheng Guan}
\date{May 30, 2018}
\begin{document}
\maketitle
\begin{abstract}
  \emph{This paper addresses the task of estimating the 6D pose of a known 3D object
  from a single RGB-D image. Most modern approaches solve this task in three steps:
   i) Compute local features; ii) Generate a pool of pose-hypotheses; iii)
   Select and refine a pose from the pool. This work focuses on the second step.
   While all existing approaches generate the hypotheses pool via local reasoning,
   e.g. RANSAC or Hough-voting, we are the first to show that global reasoning
    is beneficial at this stage. In particular, we formulate a novel fully-connected
     Conditional Random Field (CRF)thatoutputsaverysmallnumberofpose-hypotheses.
      Despite the potential functions of the CRF being nonGaussian, we give a new and
       efficient two-step optimization procedure, with some guarantees for optimality.
   We utilize our global hypotheses generation procedure to produce results that
   exceed state-of-the-art for the challenging “Occluded Object Dataset”.}
\end{abstract}
\section{Introduction}
Thetaskofestimatingthe6Dposeoftexture-lessobjects has gained
a lot of attention in recent years. From an application perspective this
 is probably due to the growing interestinindustrialrobotics,
 and invarious forms of augmented reality scenarios.
 From an academic perspective the dataset of Hinterstoisser \emph{et al.} \cite{c1}
 marked a milestone, since researchers started to benchmark
 their efforts and progress in research started to be more measurable.
 In this work we focus on the following task. Given an RGB-D image of a 3Dscene,
 in which a known 3D object is present,i.e.its3D shape and appearance is known,
  we would like to identify the 6D pose (3D translation and 3D rotation)
  of that object.

  Let us consider an exhaustive-search approach to this problem.
  We generate all possible 6D pose hypotheses,and for each hypothesis
  we run a robust ICP algorithm \cite{c2} to estimate a robust geometric
   fit of the 3D model to the underlying data. The final ICP score can then be used
   as the objective function to select the final pose. This approach has two great advantages:
   (i)It considers all hypotheses;(ii) It uses a geometric error to prune all
   incorrect hypotheses. Obviously,this approach is in feasible from a computational
   perspective, hence most approaches generate first a pool of hypotheses and use a geometrically motivated
   scoring function to select the right pose, which can be refined with robust ICP if necessary.
   Table \ref{tb1} lists five recent works with different strategies
   for “hypotheses generation” and “geometric selection”.
   The first work by Drost \emph{et al.} \cite{c3},
    and recently extended by Hinterstoisser \emph{et al.} \cite{c4},
    has no geometric selection process,and generate savery large number
    of hypotheses. The pool of hypotheses is put into a Houghspace and the peak of
    the distribution is found as the final pose.Despite its simplicity,
    the method achieves very good results, especially on the “Occluded Object Dataset”1,
     i.e. where objects are subject to strong occlusions.
     We conjecture that the main reason for its success is that it generates hypotheses
     from all local neighborhoods in the image. Especially for objects that are subject to strong occlusions,
     it is important to predict poses from as local information as possible.
     The other three approaches use triplets, and are all similar in spirit.

     \begin{figure}[htbp]
     \centering
     \includegraphics[scale=0.4]{1.png}
     \caption{Given an RGB-D input image(left) we aim at finding the 6D pose of a given object, despite it being strongly occluded (see zoom). Here our result (green) is correct, while outputs an incorrect pose (red). The key concept of this work is to have a global, and hence powerful, geometric check, in the beginning of the pose estimation pipeline. This is in stark contrast to local geometric checks performed by all other methods.In a first step, a random forest predicts for each pixel a set of three possible object coordinates, i.e. dense continuous part labeling of the object (middle). Given this, a fully-connected pairwise Conditional Random Field (CRF) infers globally those pixels which are consistent with the 6D object pose. We refer to those pixels as pose-consistent. The final pose is derived from these pose-consistent pixels via an ICP-variant}
     \label{fig1}
     \end{figure}

  \newcommand{\tabincell}[2]{\begin{tabular}{@{}#1@{}}#2\end{tabular}}
  \begin{table*}[htbp]
  \label{tb1}
  \begin{tabular}{|c|c|c|c|c|c|c|}
  \hline
  \texbf{Method} & \tabincell{c}{Intermediate\\Representation}&\tabincell{c}{Hypotheses\\Generation}&\tabincell{c}{Average Number\\of Hypotheses}&\tabincell{c}{Hypotheses\\Selection}&\tabincell{c}{Hypotheses\\Refinement}&\tabincell{c}{Run\\Time}\\
  \hline
  \hline
  Drost\emph{et al.}\cite{c3}&\tabincell{c}{Dense Point\\Pair Features}&\tabincell{c}{All local\\pairs}&20.000&\tabincell{c}{Sub-optimal\\search}&ICP&0.4s\\
  our&\tabincell{c}{multiple object\\coordinates}&\tabincell{c}{Fully-connected CRF\\with geometric check}&0-10&\tabincell{c}{Optimal w.r.t.\\ICP variant}&\tabincell{c}{ICP\\variant}&1-3s\\
  \hline
  \end{tabular}
  \caption{A broad categorization of six different 6D object pose estimation methods with respect to four different computational steps: (a) Intermediate representation, (b) Hypotheses generation, (c) Hypotheses selection, (d) Hypotheses refinement, (e) Runtime.}
  \end{table*}

   In a first step they compute for everypixelone,or more,
   so-called object coordinates,a 3D continuous part-label on the given
   object (see Fig. \ref{fig1} right). Then they collect locally triplets of points,
   in [33] these are all local triplets and  they are randomly sampled with RANSAC.
   For each triplet of object coordinates they first perform a
   geometry consistency check , and if successful,
   they compute the 6D object pose, using the Kabsch algorithm.
\bibliographystyle{plain}
\bibliography{6D}
\end{document}

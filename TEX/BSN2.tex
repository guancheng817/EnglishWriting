\documentclass[10pt,twocolumn,letterpaper]{article}
\usepackage{cvpr}
\usepackage{times}
\usepackage{epsfig}
\usepackage{graphicx}
\usepackage{amsmath}
\usepackage{amssymb}
\usepackage{times}
\usepackage{stfloats}
\usepackage[pagebackref=true,colorlinks,linkcolor=red,citecolor=green,breaklinks=true,bookmarks=false]{hyperref}
\cvprfinalcopy
\def\cvprPaperID{****} % *** Enter the CVPR Paper ID here
\def\httilde{\mbox{\tt\raisebox{-.5ex}{\symbol{126}}}}
\title{BSN: Boundary Sensitive Network for Temporal Action
	Proposal Generation}
\author{Cheng Guan\\\\
August 19, 2018}
\begin{document}
\maketitle
\section{Approach}
\subsection{Problem Definition}
An untrimmed video sequence can be denoted as $X = {\left\lbrace x_n\right\rbrace}_{n=1} $
with $l_v$ frames, where
$x_n$ is the $n$-th frame in $X$. Annotation of video $X$ is composed by a set of action in-
$N_g$
stances $\psi_g = {\left\lbrace \varphi_n = \left(t_{s,n},t_{e,n}\right)\right\rbrace}_{n=1}^{N_g}$
, where $N_g$ is the number of truth action instances
in video $X$, and $t_{s,n},t_{e,n}$ are starting and ending time of action instance $\varphi_n$  separately.
Unlike detection task, classes of action instances are not considered in temporal action proposal generation.
\subsection{Video Features Encoding}
To generate proposals of input video, first the authors need to extract feature to encode visual
content of video. In their framework, they adopt two-stream network \cite{simonyan2014two} as visual en-
coder, since this architecture has shown great performance in action recognition task
\cite{wang2016temporal} and has been widely adopted in temporal action detection and proposal generation
tasks \cite{zhao2017temporal,gao2017cascaded}. Two-stream network contains two branches: spatial network operates
on single RGB frame to capture appearance feature, and temporal network operates on
stacked optical flow field to capture motion information.
\par
\begin{figure*}[b]
	\centering
	\includegraphics[width=1\linewidth]{1.png}
	\caption{Details of proposal generation module. First, to generate
		candidate boundary locations, the authors choose temporal locations with high boundary prob-
		ability or being a probability peak. Then, they combine candidate starting and ending
		locations as proposals when their duration satisfying condition. Construct BSP feature. Given a proposal and actionness probabilities sequence, they can sample actionness
		sequence in starting, center and ending regions of proposal to construct BSP feature.}
	\label{fig1}
\end{figure*}
To extract two-stream features, as shown in Fig.~\ref{fig1}, first they compose a snippets
$l_s$
sequence $S = {\left\lbrace s_n \right\rbrace}_{n=1}^{l_s} $
from video $X$, where $l_s$ is the length of snippets sequence. A
snippet $s_n=\left(x_{t_n},o_{t_n}\right)$ includes two parts: $x_{t_n}$ is the $t_n$-th RGB frame in $X$ and $o_{t_n}$ is
stacked optical flow field derived around center frame $x_{t_n}$.
\section{Boundary-Sensitive Network}
To achieve high proposal quality with both precise temporal boundaries and reliable
confidence scores, the authors adopt ``\textit{local to global}'' fashion to generate proposals. In BSN, they
first generate candidate boundary locations, then combine these locations as proposals
and evaluate confidence score of each proposal with proposal-level feature.
\par
\noindent\textbf{Network Architecture.} The architecture of BSN is presented in Fig.\ref{fig1}, which contains three modules: temporal evaluation, proposal generation and proposal evaluation. \textit{Temporal evaluation module} is a three layers temporal convolutional neural network,
which takes the two-stream feature sequences as input, and evaluates probabilities of
each temporal location in video whether it is inside or outside, at or not at boundaries
of ground truth action instances, to generate sequences of starting, ending and actionness probabilities respectively.
\par
\noindent\textbf{Proposal generation module.} The goal of proposal generation module is to generate
candidate proposals and construct corresponding proposal-level feature. The authors achieve this
goal in two steps. First they locate temporal locations with high boundary probabilities,
and combine these locations to form proposals.
\par
\noindent\textbf{Proposal evaluation module.} The goal of proposal evaluation module is to evaluate
the confidence score of each proposal whether it contains an action instance within
its duration using BSP feature. Hidden layer with 512 units handles the input
of BSP feature $f_{BSP}$ with Relu activation. The output layer outputs confidence score
$p_{conf}$ with sigmoid activation,
\bibliography{BSN}
\bibliographystyle{ieee}
\end{document}
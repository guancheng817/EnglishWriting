\documentclass[12pt]{article}
\usepackage{graphicx}
\usepackage{cite}
\usepackage{CJK}
\usepackage{setspace}
\usepackage{amsmath}

\twocolumn
\begin{document}
\title{CNN(2)}
\author{Cheng Guan}
\date{May 5, 2018}
\maketitle
\setlength{\baselineskip}{20pt}

  The convolution neural network usually contains the following layers:

  \textbf{Convolutional layer}:The convolution layer of convolution neural network is composed of
 several convolution units, and the parameters of each convolution unit are optimized by backpropagation algorithm.
  The purpose of convolution operation is to extract different features of input.
  The first layer convolution layer may only extract some low-level features such as edges, lines and corners,
  and more layers can extract more complex features from the low-level features.

  \textbf{Rectified Linear Units layer, ReLU layer}:
  This layer of neural activation function  uses Rectified Linear Units .
  \begin{equation}
    f(x)=max(0,x)
  \end{equation}

  \textbf{Pooling layer}:
  The characteristics of a large dimension are usually obtained after convolutional layer,
   and the features are cut into several regions, and get their maximum or average values ,
   and a new and smaller dimension is obtained.

  \textbf{Fully-Connected layer}:
  The combination of all local features into global features
  is used to calculate the final scores of each class.\cite{test2}

  Here are the application examples of various layers of convolution neural network: picture ~\ref{fig1}

  \begin{figure}[htbp]
  \centering
  \includegraphics[width=6cm,heght=8cm]{1.png}\\
  \caption{Application Example}\label{fig1}
 \end{figure}

\bibliographystyle{plain}
\bibliography{cite20}
\end{document}

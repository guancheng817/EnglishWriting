\documentclass[10pt,twocolumn,letterpaper]{article}
\usepackage{cvpr}
\usepackage{times}
\usepackage{epsfig}
\usepackage{graphicx}
\usepackage{amsmath}
\usepackage{amssymb}
\usepackage{times}
\usepackage{stfloats}

\usepackage[pagebackref=true,colorlinks,linkcolor=red,citecolor=green,breaklinks=true,bookmarks=false]{hyperref}
\cvprfinalcopy
\def\cvprPaperID{****} % *** Enter the CVPR Paper ID here
\def\httilde{\mbox{\tt\raisebox{-.5ex}{\symbol{126}}}}
\title{Person Transfer GAN to Bridge Domain Gap for Person Re-Identification}
\author{Cheng Guan\\\\
July 26, 2018}
\begin{document}
\maketitle
\begin{abstract}
	With the development of the Re-Identification (ReID), although the performance of person Re-Identification
	has been significantly boosted, many challenging issues in real scenarios have not been fully investigated, \textit{e.g.},
	the complex scenes and lighting variations, viewpoint and
	pose changes, and the large number of identities in a camera network. To facilitate the research towards conquering those issues, this paper contributes a new dataset called
	MSMT17  with many important features, \textit{e.g.}, 1) the raw
	videos are taken by an 15-camera network deployed in both
	indoor and outdoor scenes, 2) the videos cover a long period of time and present complex lighting variations, and 3)
	it contains currently the largest number of annotated identities, \textit{e.g.}, 4,101 identities and 126,441 bounding boxes. The authors also observe that, domain gap commonly exists between
	datasets, which essentially causes severe performance drop
	when training and testing on different datasets.
\end{abstract}
\section{Introduction}
	Person Re-Identification (ReID) targets to match and return images of a probe person from a large-scale gallery
	set collected by camera networks. Because of its important
	applications in security and surveillance, person ReID has
	been drawing lots of attention from both academia and industry. Thanks to the development of deep learning and the
	availability of many datasets, person ReID performance has
	been significantly boosted. 
	\par
	Although the performance on current person ReID
	datasets is pleasing, there still remain several open issues hindering the applications of person ReID. The currently largest \textit{DukeMTMC-reID}
	 \cite{zheng2017unlabeled} contains less than 2,000 identities and presents simple lighting conditions. Those limitations simplify the person ReID task and help to achieve high accuracy. Another challenge they observe is that, there exists domain gap between different person ReID datasets, \textit{i.e.}, training and testing on different person ReID datasets results in severe performance drop. As shown in Fig.\ref{fig1} the
	 domain gap could be caused by many reasons like different
	 lighting conditions, resolutions, human race, seasons, backgrounds, \textit{etc}.
	 \begin{figure}
	 	\centering
	 	\includegraphics[width=1\linewidth]{1.png}
	 	\caption{Illustration of the domain gap between $CUHK03$
	 		and $PRID$. It is obvious that, $CUHK03$ and $PRID$ present
	 		different styles, \textit{e.g.}, distinct lightings, resolutions, human race, seasons, backgrounds, \text{etc.}, resulting in low accuracy
	 		when training on $CUHK03$ and testing on $PRID$.}
 		\label{fig1}
	 \end{figure}
 \par
 To address the second challenge, they propose to bridge
 the domain gap by transferring persons in dataset \textit{A} to an-
 other dataset \textit{B}. The transferred persons from \textit{A} are desired
 to keep their identities, meanwhile present similar styles,
 \textit{e.g.}, backgrounds, lightings, \textit{e.g.}, with persons in \textit{A}. they
 model this transfer procedure with a Person Transfer Gener-
 ative Adversarial Network (PTGAN), which is inspired by
 the Cycle-GAN \cite{zhu2017unpaired}.
 \par
 Their contributions can be summarized into three aspects. 1) \textit{A} new challenging large-scale \textit{MSMT17} dataset
 is collected and will be released. Compared with existing
 datasets, \textit{MSMT17} defines more realistic and challenging
 person ReID tasks. 2) They propose person transfer to take
 advantages of existing labeled data from different datasets.
 It has potential to relieve the expensive data annotations on
 new datasets and make it easy to train person ReID systems
 in real scenarios. An effective PTGAN model is presented
 for person transfer. 3) This paper analyzes several issues
 hindering the applications of person ReID. The proposed
 \textit{MSMT17} and algorithms have potential to facilitate the future research on person ReID.
 \section{Related Work}
 This work is closely related with descriptor learning in
 person ReID and image-to-image translation by GAN. they
 briefly summarize those two categories of works in this section.
 \bibliography{PTGAN}
 \bibliographystyle{ieee}
\end{document}
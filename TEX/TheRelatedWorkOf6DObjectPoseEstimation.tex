\documentclass[10pt,twocolumn,a4paper]{article}
\usepackage{times}
\usepackage{epsfig}
\usepackage{graphicx,float}
\usepackage{indentfirst}
\usepackage{balance}
\usepackage{cite}
\usepackage[pagebackref=true,breaklinks=true,colorlinks,bookmarks=false]{hyperref}
\usepackage{geometry}
\geometry{left=2.0cm,right=2.0cm,top=2.5cm,bottom=2.5cm}
\setlength{\parindent}{1em}
\title{The Related Work of 6D Object Pose Estimation}
\author{Cheng Guan}
\date{\today}
\begin{document}
\maketitle
\section{Related work}
  The topic of object detection and pose estimation has
been widely researched in the past decade. In the brief review below,
 we focus only on recent works and split them into three categories.
 We will omit the methods \cite{c1} since they were already discussed
 in the previous section.

 \textbf{Sampling-Based Methods.} Sparse feature based methods \cite{c2}
 have shown good results for accurate pose estimation. They extract points of
 interest and match them based on a RANSAC sampling scheme. With the shift of the
 application scenario into robotics their popularity declined since they rely on texture.
  Shotton \emph{et al.} \cite{c3} addressed the task of camera re-localization by introducing the concept of scene coordinates.
  They learn a mapping from camera coordinates to world coordinates and generate camera pose hypotheses by
   random sampling. Most recently Phillips \emph{et al.} \cite{c4} presented a method for
   pose estimation and shape recovery of transparent objects where a random forest is trained
   to detect transparent object contours. Those edge responses are clustered and random sampling
  is employed to find the axis of revolution of the object. Instead of
   randomly selecting individual pixels we will use the entirety of the image to
   find pose hypotheses.

   \textbf{Non-Sampling-BasedMethods.} An alternative to random sampling of pose hypotheses are
   Hough-voting based methods where all pixels cast a vote into a quantized
   prediction space (e.g. 2D object center and scale). The cell with the majority
   of votes is taken as the winner. Template  have also been applied to the task
   of pose estimation. To find the best match the template is scanned across
   the image and a distance metric is computed at each position. Those
    methods are harmed by clutter and occlusion which disqualifies them to be
     applied toourscenario. In our approach each pixel is processed,but
     instead of them voting individually we find pose-consistent pixel-sets
     by global reasoning.

     \textbf{Pose Estimation using Graphical Models.}In an older piece of work
     the pose of object categories was found in images either in 2D or in 3D.
     They also use the key concept of discretized object coordinates for object
      detection and pose estimation. The MRF-inference stage for finding
      pose-consistent pixels is closely related to ours. Foreground pixels are
      accepted when the layout consistency constraint (where layout consistency
      means that neighboring pixels should be long to the same part)is satisfied.
      However since the shape of the object is unknown, the pairwise terms are
      not as strong as in our case. The closest related work to ours is Bergholdt
      \emph{et al.} \cite{c5}. They use the same strategy of discriminatively
      modeling the local appearance of object parts and globally inferring the
       geometric connections between them. To detect and find the pose of
       articulated objects (faces, human spines, human poses) they extract
       feature points locally and combine them in a probabilistic,
       fully-connected, graphical model. However they rely on a exact solution to the problem
       while a partial optimal solution is sufficient in our case.
       We therefore employ a different approach to solve the task.


     \bibliographystyle{plain}
     \bibliography{RelatedWork}
      \end{document}

\documentclass[10pt,twocolumn,letterpaper]{article}
\usepackage{cvpr}
\usepackage{times}
\usepackage{epsfig}
\usepackage{graphicx}
\usepackage{amsmath}
\usepackage{amssymb}
\usepackage{times}
\usepackage{stfloats}
\usepackage[pagebackref=true,colorlinks,linkcolor=red,citecolor=green,breaklinks=true,bookmarks=false]{hyperref}
\cvprfinalcopy
\def\cvprPaperID{****} % *** Enter the CVPR Paper ID here
\def\httilde{\mbox{\tt\raisebox{-.5ex}{\symbol{126}}}}
\title{From Noise Modeling to Blind Image Denoising}
\author{Cheng Guan\\\\
July 12, 2018}
\begin{document}
\maketitle
\section{LR-MoG Filter}
In this section, the authors introduces the LR-MoG filter to
recover low-rank signals from noisy signals contaminated by
MoG noise. Let $X=\left[x_1,x_2,\cdot\cdot\cdot\right]$ be a
$d\times n$ matrix and $x_i \in \mathbb{R}^d,i\in \left(1,2,\cdot\cdot\cdot,n\right)$, be a noisy signal. X can be
decomposed as shown in Eq.~\ref{eq1}
\begin{equation}
 X = \hat{X} + E
\label{eq1}
\end{equation}
Here,$\hat{X}=\left[\hat{x_1,\cdot\cdot\cdot,\hat{x_n}}\right]$, 
where $\hat{x}_i$ is low-rank representable.$E=\left[e_1\cdot\cdot\cdot e_n\right]$, 
where $e_i$ is the noise on $x_i$ and follows MoG distribution.LR-MoG is 
used to recover $\hat{X}$ from obesered $X$.In order to get the signal 
from noise, the authors introduce Bayesian approach by introduction 
priors on $\hat{X}$ and $E$. Then ths signal is recovered via inference.
\subsection{Modeling Low-rank Component}
Firstly, the authors model the low-rank component $\hat{X}$. Decomposing 
each vector $\hat{x_i}$, where $i\in\left\lbrace 1,\cdot\cdot\cdot,n\right\rbrace$, 
with $\hat{x_i}=A\hat{y_i}+u$.A is a $d\times d$ matrix and $u$ is 
the mean vector of ${\left\lbrace \hat{x_i}\right\rbrace}_{i=1}^n$.In 
the filed of dimensional reduction \cite{nasrabadi2007pattern}, some similar decomposition 
methods can be also found. The matrix $x\hat{X}$ can be decomposed as
shown in Eq.\ref{eq2}
\begin{equation}
\hat{X} = A\hat{Y}^\mathrm{T}+U
\label{eq2}
\end{equation}
where $\hat{Y}=\left[\hat{y}_1\cdot\cdot\cdot\hat{y}_n\right]^\mathrm{T}$ 
$U=\left[u\cdot\cdot\cdot u\right]$ and the column vectors is matrix 
$A\hat{Y}$ are zero mean. The rank($M$) is the rank of a matrix $M$.
Introduce the Gaussian prior on $u$ as shown in Eq.\ref{eq3}
\begin{equation}
u \sim \mathcal{N}\left(\cdot|u_0,K_0\right)
\label{eq3}
\end{equation}
Here, $\mathcal{N}\left(\cdot|u_0,\sum \right)$ is a normal 
distribution with mean $\mu$ and covariance $\sum$. In
order to model the low-rank property of $\hat{X}$, let 
matrix $A\hat{Y}^\mathrm{T}$ be low-rank representable, 
because rank$\left(U\right)=1$ and rank$\left(\hat{X}\right)$ = 
rank$\left(A\hat{Y}^\mathrm{T}+U\right) \leq$ rank $\left(A\hat{Y}+1\right)$.
\par
To model the low-rank property of $A\hat{Y}^\mathrm{T}$, the authors
introduce the trace-norm prior \cite{candes2011robust,candes2009exact}and it can be well approximated 
by the ARD[27]. And the authors adopt the soft trace-norm prior on 
$A\hat{Y}^\mathrm{T}$ as shown in Eq.\ref{eq4} and Eq.\ref{eq5}
\begin{equation}
p\left(A\right) \varpropto {\rm exp}\left(-\frac{1}{2}{\rm tr}\left(AC_A^{-1}A^\mathrm{T}\right)\right)
\label{eq4}
\end{equation}
\begin{equation}
p\left(\hat{Y}\right) \varpropto {\rm exp}\left(-\frac{1}{2}{\rm tr}\left(\hat{Y}C_{\hat{Y}}^{-1}A^\mathrm{T}\right)\right)
\label{eq5}
\end{equation}
where ${\rm tr}\left(\cdot\right)$ represents the trace of a matrix. $C_A$ and $C_{\hat{Y}}$ are set to be diagonal positive semidefinite with Eq.\ref{eq6} and 
Eq.\ref{eq7}
\begin{equation}
C_A = diag_d \left\lbrace c^2_{a_1},\cdot\cdot\cdot,c^2_{a_d}\right\rbrace 
\label{eq6}
\end{equation}
\begin{equation}
C_{\hat{Y}} = diag_d \left\lbrace c^2_{y_1},\cdot\cdot\cdot,c^2_{y_d}\right\rbrace 
\label{eq7}
\end{equation}
Here, diag $diag_d \left\lbrace c^2_{a_1},\cdot\cdot\cdot,c^2_{a_d}\right\rbrace$ represents a $d \times d$ diagonal matrix
with diagonal items $c_1^2,\cdot\cdot\cdot,c^2_n$. Let $\hat{a}_j$ be the $j$th row vector of matrix $A$, and derive the specific formula of Eq.\ref{eq5} 
with Eq.\ref{eq8}
\begin{equation}
\begin{aligned}
p\left(A\right) &\varpropto {\rm exp}\left(-\frac{1}{2}{\rm tr}\left(AC_A^{-1}A^\mathrm{T}\right)\right) \\
& \varpropto \prod_{j=1}^d {\rm exp}\left(-\frac{1}{2c^2_{a_j}}\hat{a}_j^\mathrm{T}\hat{a}_j\right)\\
& \varpropto \prod_{j=1}^{d} \mathcal{N}\left(\hat{a}_j|0,c^2_{a_j}I\right)
\end{aligned}
\label{eq8}
\end{equation}
\section{Modeling MoG Noise}
To model the complex noise on practical noisy images,
the authors use MoG distribution for noise modeling. Because the
number of components is not provided, they use Bayesian
nonparametric technique and introduce the Dirichlet pro-
cess prior to the MoG. Each Gaussian component is repre-
sented by $\mathcal{N}\left(\cdot|\mu_i,\sum_{i}\right)$ with 
mean vector $\mu_i$ and covariancematrix $\sum_{i}$.

{\small
	\bibliographystyle{ieee}
	\bibliography{BID}
}
\end{document}
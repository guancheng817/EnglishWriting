\documentclass[10pt,twocolumn,a4paper]{article}
\usepackage{graphicx}
\usepackage{amsmath}
\usepackage{amssymb}
\usepackage[colorlinks,
            citecolor=green
            ]{hyperref}
\usepackage{setspace}
\renewcommand{\baselinestretch}{1.0}
\bibliographystyle{plain}
\setlength{\parindent}{2em}
\title{Processing Graphs With Neutral Networks}
\author{Cheng Guan}
\date{\today}
\begin{document}
\maketitle
\section{Processing Graphs With Neutral Networks}
We now describe a deep neural network suitable for processing the question
and scene graphs to infer an answer. See Fig.~\ref{fig1} for an overview.
The two graphs representing the question and the scene are processed independently in a recurrent architecture. We drop the exponents S and Q for this paragraph as the same procedure applies to both graphs.
Each node $x_i$ is associated with a gated recurrent unit
(GRU \cite{c1}) and processed over a fixed number \emph{T} of iterations (typically \emph{T}=4):
\begin{equation}
h_i^0 = 0
\end{equation}
\begin{equation}
n_i=pool_j \left ({e_{ij}}' \circ {x_j}' \right)
\end{equation}
\begin{equation}
h_i^t=GRU\left( h_i^{t-1},\left [{x_i}';n_i \right] \right)
\end{equation}\\
Square brackets with a semicolon represent a concatenation of vectors,
and $\circ$ the Hadamard (element-wise) product. The final state of the GRU is used as
 the new representation of the nodes: ${x_i}"=h_i^{\emph{T}}$ . The pool operation transforms features from a variable number of neighbours (i.e. connected nodes) to a fixed-size representation. Any commutative operation can be used (e.g. sum, maximum). In our implementation, we found the best performance with the average function, taking care of averaging over the variable number of connected neighbours. An intuitive interpretation of the recurrent processing is to progressively integrate context information from connected neighbours into each node’s own representation.
 A node corresponding to the word ’ball’, for instance, might thus incorporate
 the fact that the associated adjective is ’red’. Our formulation is similar but
 slightly different from the gated graph networks \cite{c2},as the propagation of information in our model is limited to the first order.
  Note that our graphs are typically densely connected.

  We now introduce a form of attention into the model,
  which constitutes an essential part of the model. The motivation is two-fold:
  (1) to identify parts of the input data most relevant to produce the answer
  and (2) to align specific words in the question with particular elements of the scene.
  Practically, we estimate the relevance of each possible pairwise combination of words and objects.
   More precisely, we compute scalar “matching weights” between node sets $\left \{ x^{'Q} \right \}$ and $\left \{ x^{'S} \right \}$.
   These weights are comparable to the “attention weights” in other models \cite{c3}.
   Therefore, $\forall i \in 1...N^Q,j\in 1...N^S$:
   \begin{equation}
    a_{ij}=\sigma \left(W_5\left ( \frac{x_i^{'Q}}{\left \| x_i^{'Q} \right \|} \circ \frac{x_i^{'Q}}{\left \| x_i^{'Q} \right \|} \right) + b_5\right)
   \end{equation}

   where $W_5 \in \mathbb{R}^{1\times h}$ and $b_5 \in \mathbb{R}$ are learned weights and biases,
   and $\sigma$ the logistic function that introduces a nonlinearity and bounds the weights to (0,1).
   The formulation is similar to a cosine similarity with learned weights on the feature dimensions.
    Note that the weights are computed using the initial embedding of the node features (pre-GRU).
    We apply the scalar weights $a_{ij}$ to the corresponding pairwise combinations of question and
    scene features, thereby focusing and giving more importance to the matched pairs (Eq.\ref{eq1}).
     We sum the weighted features over the scene elements (Eq.\ref{eq2}) then over the question elements (Eq.\ref{eq3}),
  interleaving the sums with affine projections and non-linearities to obtain a final prediction:

  \begin{equation}
  y_{ij}=a_{ij}\cdot \left[x_i^{''Q};x_j^{''S} \right]
  \label{eq1}
  \end{equation}
  \begin{figure*}[htbp]
  \centering
  \includegraphics[scale=0.5]{1.png}
  \caption{ Architecture of the proposed neural network. The input is provided as a description of the scene (a list of objects with their visual characteristics) and a parsed question (words with their syntactic relations). The scene-graph contains a node with a feature vector for each object, and edge features that represent their spatial relationships. The question-graph reflects the parse tree of the question, with a word embedding for each node, and a vector embedding of types of syntactic dependencies for edges. A recurrent unit (GRU) is associated with each node of both graphs. Over multiple iterations, the GRU updates a representation of each node that integrates context from its neighbours within the graph. Features of all objects and all words are combined (concatenated) pairwise, and they are weighted with a form of attention. That effectively matches elements between the question and the scene. The weighted sum of features is passed through a final classifier that predicts scores over a fixed set of candidate answers.}
  \label{fig1}
 \end{figure*}
  \begin{equation}
  y_i^{'}=f \left( W_6\sum_{j}^{N^S}y_{ij}+b_6\right)
  \label{eq2}
  \end{equation}
  \begin{equation}
  y_i^{''}=f^{'}\left( W_7\sum_{i}^{N^Q}y_i^{'}+b_7\right)
  \label{eq3}
  \end{equation}
  with $W_6,W_7,b_6,b_7$ learned weights and biases, $f$ a ReLU, and $f^{'}$ a softmax
  or a logistic function . The summations over
  the scene elements and question elements is a form of pooling that brings the
   variable number of features (due to the variable number of words and objects in the input)
    to a fixed-size output. The final output vector $y^{''}\in \mathbb{R^{\emph{T}}}$ contains scores
     for the possible answers, and has a number of dimensions equal to 2 for the binary questions
     of the “balanced” dataset, or to the number of all candidate answers
     in the “abstract scenes” dataset. The candidate answers are those appearing at least 5 times
      in the training set (see supplementary material for details).

   \bibliography{graphCite2}
   \end{document}

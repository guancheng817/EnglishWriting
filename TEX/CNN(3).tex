\documentclass[12pt,twocolumn]{article}
\usepackage{graphicx}
\usepackage{setspace}
\usepackage{cite}
\usepackage{CJK}
\usepackage{indentfirst}
\setlength{\parindent}{2em}
\setlength{\parskip}{1em}
\setlength{\baselineskip}{20pt}
\begin{document}
\title{Convolutional Layer--Local Connectivity}
\author{Cheng Guan}
\date{May 9, 2018}
\maketitle
\section{Local Connectivity}

  The general neural network designs the input layer and the hidden layer
  by "Full Connected". From the point of view of computation,
  it is feasible to calculate features from the whole image of relatively small images.
  However, from a computational point of view,if it is a larger image (such as a 96x96 image),
  it will be time-consuming to learn the features of the entire image through this approach .
  A simple way to solve such problems in the convolution layer is to limit the connection
  between the hidden elements and the input units: each hidden element can only be connected only to a part of the input unit.
  For example, each hidden unit connects only a small piece of adjacent area of the input image.
  The size of the input region connected to each hidden unit is called the receptive field of the r neuron.

  As shown in the following figure ~\ref{fig1}, the size of the sample input unit is 32 x 32 x 3,
  the depth of the output unit is 5, and the same location for the different depth of the output unit is the same as the area connected to the input picture,
  but the parameters (filters) are different. \cite{test2}

 Although each output unit is only part of the connection input,
 the calculation method of the value is not changed, both the weight and dot product of the input,
 and add then the offset, which is the same as that of the ordinary neural network, as shown in the following figure:~\ref{fig2}

 \begin{figure}[ht]
 \centering
 \includegraphics[width=5cm]{1.png}\\
 \caption{Local Connectivity}
 \label{fig1}
 \includegraphics[width=5cm]{2.png}\\
 \caption{Method Of Calculating}
 \label{fig2}
\end{figure}


\bibliographystyle{plain}
\bibliography{cite20}
\end{document}

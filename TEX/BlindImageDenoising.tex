\documentclass[10pt,twocolumn,letterpaper]{article}
\usepackage{cvpr}
\usepackage{times}
\usepackage{epsfig}
\usepackage{graphicx}
\usepackage{amsmath}
\usepackage{amssymb}
\usepackage{times}
\usepackage{stfloats}
\usepackage[pagebackref=true,colorlinks,linkcolor=red,citecolor=green,breaklinks=true,bookmarks=false]{hyperref}
\cvprfinalcopy
\def\cvprPaperID{****} % *** Enter the CVPR Paper ID here
\def\httilde{\mbox{\tt\raisebox{-.5ex}{\symbol{126}}}}
\title{From Noise Modeling to Blind Image Denoising}
\author{Cheng Guan\\\\
July 10, 2018}
\begin{document}
\maketitle
\begin{abstract}
As we know, traditional image denoising algorithms always consider
the noise as  white Gaussian noise.
However, in the real world, the noise on real images can be more com-
plex actually. In this paper, in order to address this problem, the authors
proposes a new blind image denoising algorithm. The algorithm can deal with real-world noisy images even when the noise model is not providedI. Using the mixture model of Gauss distribution (MOG) to model the image noise, it can approximate a wide range of continuous distribution. In this paper, the authors use Bayesian nonparametric technique
and proposes a novel Low-rank MoG filter (LR-MoG) to
recover clean signals (patches) from noisy ones contami-
nated by MoG noise.
\end{abstract}
\section{Introduction}
In the field of computer vision and image processing, image denoising
is always an much important problem. For example, we can process a
image to get high-quality image and it is an important pre-processing step
for many high-level visual problems such as digital entertainment, 
object recogniton, image segmentation, remote sensing imaging and so on. The
purpose is to recover the clean image $\hat{X}$ from its noisy 
observation $X$ which is contaminated by noise $E$ as shown in Eq.\ref{eq1}.
\begin{equation}
X = \hat{X} + E
\label{eq1}
\end{equation}
\begin{figure*}
	\centering
	\includegraphics[scale=0.4]{1.png}
	\caption{From left to right: an old photo ``Pele'' with noise; the denoising result of \cite{lebrun2015multiscale} and ours. Zoom in for better visualization.
	}\label{fig1}
\end{figure*}
Estimating $\hat{X}$ from $X$ is an inverse problem. In recent years, 
many scholars proposed some algorithms \cite{starck2002curvelet,burger1996structure,dabov2007image,buades2005non,elad2006image,lebrun2015noise} to
solove the problem. However, most methods consider the noise as the
homogeneous white Gaussian distribution. This assumption seems reasonable,
because some noise can be converted to Gaussian noise. However, in 
camera systems, the noise has many sources and can be more complex.
So the authors proposes the ``blind image denoising'' algorithm to estimate 
the noise model from a noisy observation.
\par
The traditional noise models can  fail to solve empirical
noise. Fig.~\ref{fig1} demonstrates the performance of \cite{lebrun2015multiscale} on an
old photo ``Pele''. Because of lack of flexibility of the noise
model, \cite{lebrun2015multiscale} can only remove some noise but not all noise
in some areas and over-smooth details in some other areas.
To tackle the problem, the authors of this paper proposed a new non-local 
blind image denoising algorithm.
\section{Background}
In this section, the authors introduces the information of Dirichlet process and its
construction, which will be used to construct the LR-MoG filter.
\subsection{Dirichlet Process}
We know that the Dirichlet Process (DP) is a distribution over distribu-
tions, \textit{i.e.}, each draw from a DP is itself a distribution. The
DP is parameterized by a base distribution $H$ and a concentration parameter $\alpha$. A good feature favored by DP is that a
drawn $G$ from a DP is discrete with probability one and
the dimension of $G$ is infinite.
\subsection{Stick-breaking Construction}
The stick-breaking construction is one of the
methods to explicitly represent draws from DP. With a stick-breaking construction, one can directly work with $G$ before drawing $\theta$. Sethuraman proved that a draw $G$ from
DP $\left(α,H\right)$ can be described as Eq.\ref{eq2}
\begin{equation}
\begin{aligned}
&v_i \sim \textbf{Beta}\left(1,\alpha \right), &\pi \sim v_i \prod_{j=1}^{i-1},\\
&\theta_i \sim H, &G = \sum_{i=1}^{\infty}\pi_i \delta_{\theta_i}
\end{aligned}
\label{eq2}
\end{equation}
Here, $\textbf{Beta}\left(a,b\right)$ is a \textbf{Beta} distribution with parameters a
and $b$. And $\delta_{\theta_i}$ is the Dirac probability measure concentrated
at $\theta_i$. $\pi$ are the stick lengths, and it is almost sure that
$\sum_{i=1}^{\infty}\pi_i = 1$. The stick-breaking construction indicates the
discreteness of $G$ as well.
{\small
	\bibliographystyle{ieee}
	\bibliography{BID}
}
\end{document}

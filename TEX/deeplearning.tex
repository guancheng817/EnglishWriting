\documentclass[12pt]{article}
\usepackage{cite}
\usepackage{CJK}
\usepackage{graphicx}
\usepackage{setspace}

\twocolumn
\begin{document}
\title{Deep Learning}
\author{Cheng Guan}
\date{April 29, 2018}
\maketitle
\setlength{\baselineskip}{20pt}

    The concept of deep learning stems from the research of artificial neural network.
    .The observed values (such as an image) can be represented in a variety of ways,
    such as a vector of each pixel intensity value, or more abstractly represented as a series of edges, specific shapes, and so on. Using some specific representation is easier to learn tasks from examples, such as face recognition or facial expression recognition.

  For deep learning, the idea is to stack multiple layers, that is to say, the output of this layer is the input of the next level. In this way,
  the input information can be classified and expressed.

   Deep learning allows computational models that are composed of multiple processing layers to learn representations of data with multiple levels of abstraction.
  These methods have dramatically improved the state-of-the-art in speech recognition, visual object recognition,
  object detection and many other domains such as drug discovery and genomics.\cite{test2}
  The following picture ~\ref{fig1}.

 \begin{figure}[ht]
 \centering
 \includegraphics[width=5cm,height=5cm]{DP.png}
 \caption{Deep Learning}
 \label{fig1}
\end{figure}


\bibliographystyle{plain}
\bibliography{cite2}
  \footnote{\centering From China Daily}
\end{document}
